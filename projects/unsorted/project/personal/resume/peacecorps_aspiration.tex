\documentclass[12pt,a4paper,twoside]{article}  % Comments after  % are ignored
\usepackage{amsmath,amssymb,amsfonts}          % Typical maths resource packages
\usepackage{hyperref}                          % For creating hyperlinks in cross references
\usepackage{fullpage}

\def\author{Will Holcomb}
\def\depdate{June 28, 2003}
\def\head{\author\ (Departing \depdate)}

\usepackage{fancyhdr}  % for page margins and headers 
                       % (upgraded version of fancyheadings) 
\pagestyle{fancy}
%\fancyhead{}                  % clear all fields
\fancyhead[RO,LE]{\thepage}   % RO right odd, LE left even: page number
\fancyhead[RE,LO]{\head}

% The \@ macros are internal to latex and not accessible here
%  without fooling around a bit. This does that; though I don't
%  know why...
% This simply removes the section numbering. The original was:
%  \def\@seccntformat#1{\csname the#1\endcsname\quad}
\makeatletter
\renewcommand*{\@seccntformat}[1]{}
\makeatother

\begin{document}

\thispagestyle{empty} % no header and page numbering on first page

\begin{center}
{\bf \Huge Aspiration Statement} \\
{\bf \author} \\
{\bf Serving in Mauritania, Africa} \\
{\bf Departing \depdate}
\end{center}

\section{Expectations:}

In general, information about Mauritania is not very forthcoming. As
such my expectations about the nature of my service are fairly
limited.

Things that I am fairly certain of are:

\begin{itemize}
\item I will be in Mauritania, Africa for approximately 27 months
\item It will be hot and there will lots of Islamic believers
\end{itemize}

I have only very general impressions of life in countries less
economically developed that the United States. I expect disease and
hunger to be much more prevalent. I expect the general values of the
people I meet to be shifted some in dealing more often with
fundamental survival issues. As for the specifics, I figure that I
will learn all I want to know soon enough.

Concern over my assignment details was a frustrating aspect of the
decision to commit. From the information I received it appears that
the program has two main focuses; business development and
technological development. I am a technologist, and a good one at
that. I am honestly hoping for an assignment where my skills can be
put to use.

\section{Cultural Adaptation Strategies:}

One of the most intimidating aspects of moving to Mauritania for me is
the change in language. I enjoy language and am accustomed to being
able to express myself. The thought of not being able to get the
thoughts out of my head because I don't know the words is very
intimidating to me. I have started in with some books and a computer
program in the hopes of getting a better grasp on French before
arriving.

An issue that I saw mentioned in the letters was religious
proselytizing. I live in one of the more conservatively religious
sections of the country. People trying to persuade me to a particular
belief system is not a new experience nor one that I find especially
objectionable.

I spent a year while in school working in a town a couple hundred
miles away from all of my friends. It was challenging both in that it
was a new place and that everyone that I was working with was
significantly older than I was. I found that while there the most
useful things that I could do to cope with the loneliness were to form
relationships in the place that I was and to find activities to occupy
myself. I hope to manage some analogue of these in dealing with the
transition to Mauritania.

\section{Personal and Professional Goals:}

My lack of knowledge about the needs of developing countries is a real
limitation in developing a clear set of goals for my time in
Mauritania. As I mentioned before, I believe I have a particularly
useful set of skills as a technologist and I hope to use them.

That said, I want to try and address problems that are at the level of
the needs of the community. For instance, if there a real need for
(and I can be useful in attaining) drinking water then I would want to
work on that rather than helping to develop network
infrastructure. Even though I have some solid experience designing
networks, and I'd probably only be of average skill in dealing with
water issues.

Put most simply, I want for my time in Mauritania to be well spent and
productive. Leaving the United States means putting off some important
aspects of my life in the hopes of doing some good. I hope for this
time to be meaningful both personally and within the community where I
am working.

I will be returning to continue my education, so if it is possible I
hope to stay somewhat abreast of the state of the world. As with most
of my other goals I don't really have enough information to know how
reasonable this is. Overall I am flexible in what I hope to achieve
and only have very general concepts such as meaningfulness and
productivity that I am particularly attached to.

\end{document}
