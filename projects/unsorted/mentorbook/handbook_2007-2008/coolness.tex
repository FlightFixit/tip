\chapter{Coolness}

\begin{authors}
	Scott Kiskaddon
\end{authors}

I propose a model of social development. 

\begin{enumerate}
   \item Instinctive
   \item Personality malleable
   \item Personality grounded
   \item Identity crisis
   \item Personality permeable
   \item Coolness
\end{enumerate}

 

These are not discrete stages, rather landmarks on a continuous pathway to coolness.

So we must ask ourselves then, what is coolness?

Coolness is being aware of one's social environment and pushing (redefining little by little) the borders of what is acceptable by means of inner-integrity. In other words, your intuition when trained and experienced at being socially adept, will decide what is cool for you. After much social development, through direct experience, a cool person does not focus on being socially adept anymore, rather he focuses on being cool, "being himself." This is only after much experience, does he no longer need to focus on the borders of what is acceptable socially. He is so much himself now, so in tune with society�s norms, that he is coolness itself. He settles down with his own personal "style," very similar to stage 3, but he is open to changing that style when he feels he is not cool anymore, which can be quite often. He keeps an ear to the norms of society and adapts almost instinctively into the edge of them, as a way of being himself.

Let me begin by delving into my social development model. 
\begin{enumerate}
   \item Instinctive: self explanatory. The personality type of a baby, animalistic, natural.
   \item Personality malleable: The personality type of a toddler, young child, and adolescent. Society shapes the young one, and they are trying to figure out what is "normal." It is not through choice, but by imitation, peer pressure, and obedience.
   \item Personality grounded: This is the personality stage we are all familiar with. A lot of people stop here, thinking that they've found their style and whatever happens, that style will never change. It's my style, it defines who I am. Take the mullet for example. People continue to keep this awful haircut alive because they are in stage 3, and do not know how to adapt with the times. They believe that their style is just right for them, which can lead to completely uncool people. (little kids with mullets. Omg!)
   \item Identity crisis: realizing that your style that you've had your whole life just won't work for you anymore. This means that your personality style is no longer in style with the times and you are called upon to change. An example would be the 80's transitioning into the 90's. How many people had to give up hot pink leotards because everyone thought that they looked stupid? Or short shorts? It's realizing that times change, and so must one's style eventually.
   \item Personality permeable: A few people make it to stage 5, seeing that styles change with the times and that society once again governs their style in a way. It's a stage of near-constant learning and development. Society governs once again what kind of style is in style, and you adapt accordingly again, only this time you are more forgiving of yourself, and less dependent upon a style that never changes. When the style does change, there is no problem adapting to it.
   \item Coolness: The subject of this paper, and the epitome of social development. It's one step beyond personality permeable, keeping the same constant flow of changing styles in society, and at the same time creating your own flow within that flow.
\end{enumerate}
 

I'm talking about (1)fashion, (2)slang, (3)body language, (4)cool actions, (5)conversational ability, (6)speech (that includes vocal quality. See the movie "My Fair Lady"), and above all this (7)integrity (that is the integration of the previous six aspects of coolness into a single way of being). 

It�s almost the same subject matter as acting, but on a much grander scale with the flow of fashion sewed into place.

It seems that there is no way that coolness can be taught, or that certain techniques can be passed on in a way that is useful because it is clear that styles change.

Actually, there are a handful of basics that will always be cool by virtue of what they are.

A cool person use some or all of these depending on their particular style. These apply to the basics of coolness I described above in the bolded sentence. 
\begin{enumerate}
   \item Variety
   \item Relaxation
   \item Simplicity
   \item Purity
   \item Timing
\end{enumerate}
 

Often, cool people specialize in one of these things, making it the center of their style almost, with a few of the others orbiting around the center. Comedians often specialize in timing, for example, they are cool because their conversational ability is timely. Variety is secondary. Or perhaps it would be the other way around, but my point is that one is emphasized over the other, there is a primary basic to cool people, where other basics are secondary.

These five things can easily be taught through education, and is. One requires a personal mentor though. I don't believe it can be taught very well from a book.

But maybe it can. I don't know.

Anyway, I hope this helps some of you Mentor people. Cheers!
