\chapter{Journaling (Quinton, Erik)}

\begin{authors}
	Quinton Westrich and Erik Hoy
\end{authors}

\section{Who's Gonna Read This?}

A core practice of Mentor is journaling. Most often when we journal, we email the journal to the entire Mentor group, or some subset of that group, e.g.\ the Minnows listserve. Sometimes we might just send them to Connie or some select friends. And sometimes, especially at the beginning, we just save the journal for ourselves until we feel more comfortable sending it to others.


\section{Why Journal?}


A question that many Mentor students ask is simply, "Why journal,
whats the point?" It is often seen as a waste of time and energy. "How
is writing down what I did today going to help me at all... etc" To
start with, journals can be more than just a list of events or a daily
habit. Generally what one puts into one's journals determines what one
gets out of them. Journals are an excellent tool for self-discovery
and understanding and can be used for a variety of reasons.


\subsection{Mapmaking: Documenting the Journey}

for lifespan contextualism

In one sense, journaling is much like keeping a dairy. It is a record
of one's current mindset, beliefs, and issues. One can come back to a
journal after years of grow and see were he or she was at the time.
With enough journals, it becomes possible to track one's progress over
the years and create a sort of inner map of where one has been. With
such a map, it even becomes possible see where one is going, and where
one could have gone.


\subsection{Lending a Hand to Others}

group support

When a member of a group shares a part of themselves, it generally
tends to bring the groups closer together if the rest of the group is
inclined to help. By writing journals and responding to those of
others, one encourages this bonding. A two way exchange is the ideal
here, but it is not always reality. Still, two-way journaling is an
excellent way to connect with the group. These connections are
extremely important when one comes across a personal problem. By
journaling to the "listserv," one lets the rest of Mentor know about
the problem and invites their consul and advice. As one's personal
path can be rough at times, it is often very helpful to have
supportive friends who understand one' s troubles and can offer help.
It is even possible that one's own troubles might end up helping
others in the with similar problems they may not have realized they
had!

\vspace{0.5cm}
\begin{center}
\setlength{\fboxrule}{1.5pt}
\fcolorbox{blue}{yellow}{
	\noindent\rule[2mm]{0.25\linewidth}{1pt}
	\textsf{JOURNAL}\rule[2mm]{0.25\linewidth}{1pt}
} \\[0.3cm]
\begin{changemargin}{0.5cm}{0.5cm}

 Somehow, it hit me at all once this morning.  I woke up around five and just started crying for no reason.
 
I'm taking 19 hours this semester.  I don't know why I thought that was a good idea.  I guess I thought I'd feel better being busy.  Anyway, that's six classes.  I'm pretty sure I'm going to fail two of them miserably.  Two more I'll barely scrape by in.

What scares me most is that I really don't care at all.  I'm not happy being here.  I feel like all I'm doing right now is wasting my time.  I don't have the motivation to do anything.  There are things I do because I feel like I should be doing them, but there's no passion in it.

And because my GPA is pretty much hitting the floor after this semester, I'll be losing a lot of scholarship money.  I can't pay for this on my own.
So...  what do I do?  I've spent most of my life busting my ass to so I could go to college and do well for myself.  I never made a plan B because I never thought I would need one.

And how do I explain this to my dad?  How can I tell him that his daughter is miserable and wants more than anything to just quit and go home?
Have any of you felt like this?  Any advice is welcomed.

\begin{flushright}
	Heather Sisk, MBTI? (2008)
\end{flushright}

\mbox{}\hfill * \hfill * \hfill * \hfill\mbox{}

*hugs*

Number one: remember, things will work out.  They will not go as you planned, or in any way you expected them to, but they will work out and you will be better.

If you will make one final push for the end of the semester, then the situation is not hopeless.  Go to the financial aide office and tell them what's going on, find out which of your scholarships have probation periods and if you can get them back if you can get your grades back up.  For each of the classes you're failing in find someone who can tutor you or help you, if only to do well on the final which can raise a grade by a letter or more depending on the class.  If you're an N, find an N to help you; if your an S, find an S to help you.  What classes are you in trouble in?  Post a list and see if there is anyone in mentor who can help.  And go to each of the professors for advice and help and to find out what your actually standing in the class is, they are usually very nice one on one.

Remember that you are not defined by the grades that you make or how well you do in school.  Do this because you want to.  Whatever happens you'll be okay.

prayers for comfort and courage for you,

\begin{flushright}
	Josh Nikkel, INTP (2008)
\end{flushright}

\end{changemargin}
\end{center}


\subsection{Dialogue}

philosophy and critical thinking, others help keep us honest with ourselves


When one is willing to place one' s beliefs on the line, the true test
begins. In Mentor, one tends to growth into many different positions
and beliefs. These are often put in journals, and this often results
in spirited discussions between those who hold different beliefs and
even sometimes with those who hold the same beliefs at a different
level. These discussions are an essential part of self-examination as
they ask one to examine one's beliefs in the context of friendly
discussion. Many times a single poi ant observation (even a seemingly
minor one) by a peer can prove to be extremely important to the growth
and development of another.


\subsection{Community}

the sangha: a safe place inviting openness, honesty, and acceptance

As noting before, journaling builds community. Mentor is a safe and
confidential place to express oneself. It is a place where anyone can
say whatever they need to and still find acceptance. The community of
Mentor is open and willing to receive. Journaling is an excellent way
to express oneself within this community. In many ways, journaling is,
in itself, a path to openness. By being open with Mentor, one starts a
pattern that will likely make one more open with oneself. Openness
breeds more openness. It is how relationships grow. An all too common
story is that of friends, spouses, or relatives who simply stopped
listening to each other and in doing so fell apart. Journaling in
Mentor helps to keep the lines of understanding and communication open
and accepting. One is not to be condemned for what is in one's heart.


\section{What do I write about?}

Journals can be about anything. The most important part is to start writing!
More often than not you'll find that starting on a mundane topic can
lead to some interesting topic. One of the
mysteries of writing is that it brings things out to the surface so
that you learn things about yourself you might not have seen otherwise.
Whatever you want: politics, daily life, philosophy, figuring
things out, religion, art, music, books, responses to mentor reading
and exercises


Intersperse with examples

\section{Quotes}

\section{Works Cited}

\section{Further Reading}

\bibliographystyle{mla}		% the style you want to use for references.
\bibliography{mr,refs}				% the files containing all the articles and books you ever referenced.
