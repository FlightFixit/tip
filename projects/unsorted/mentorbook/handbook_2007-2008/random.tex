\chapter{Random (Everyone)}

\begin{authors}
	Lots of People
\end{authors}

\section{Craig's Stuff}

New students
-actively asking big questions (seeking understanding)
		-universal scale
			-Why do bad things happen to good people?
			-What is love?
			-What is humanity?
			-What is reality?
		-people and relation
			-Why do others not always see like I do?
			-What are they seeing?
			-What if we disagree?
			-What is humanity?
		-self and identity
			-who am I?
			-Where do I fit in?

Working with older students (observing)
	-others who have been working on similar things for awhile
	-others who have been working on different things
	-quality difference of thought process
Other beginning students (observing)
-different personalities and backgrounds tackling the same questions from 
different angles

Skills and Goals

Searching
Observation
Valuation
Striving
Integration
Refinement

Hero's Journey? lol 
Pattern recognition
Detachment and Choice
Assumption recognition (and other CT skills)
Assumption evaluation
Repeat Search cycle
Fluid Identity
Pathiness 
Model Knowledge
Model Exclutions and other shades of grey
Uncertainty

\section{Scott Kiskaddon journal April 22, 2008}

	Seeking oneness can teach you a lot. Meditating and purifying oneself is a good thing.
Manyness is also an incredible teacher, not just oneness. What I mean is, try looking at the world in different ways. Perspective is powerful.
For instance, you can look at a tree and notice the oneness. If you concentrate on the connection it has to the rest of nature, the boundaries of the tree seem to dissolve. Every place in that tree is exchanging something with another place. It happens so omnisciently, that you have to wonder after a while where the tree ends and the rest of the world begins. The edges of what is considered a tree is constantly taking things in and moving things out, constant transformation, passing around what has been passed around since the dawn of time. This is the stuff that the cosmos was made from! Focusing on the oneness of the tree can teach you a lot.
And on the other hand, you can look at a tree and notice the manyness. There are just so many differences to the tree, variety. No one leaf is the same, no crag in the bark is the same. It's like an ocean of marbles, where no two marbles are alike. You can look forever at the tree, and you'd never find all of the diversity. A tree is an amazing thing, so incredibly varied. You could never get bored. There is so much to the tree, it's overwhelming. Manyness can humble you and also teach you a lot.
I think it can teach you a lot to practice both of these things at appropriate times in your life.

Mindfulness as manyness. My idea of mindfulness is to be receptive. Part of this is knowing that I can't notice everything. Infinite is infinite.
Likewise, the idea of looking for things to notice, to me, is contrary to what noticing really is. If you have to look for it, then you aren't paying attention to what's here and now, but that still doesn't discount what people find from looking. It's still a part of being receptive: you are inclusive to exclusivity.
You just take in what you can.
I think it helps to have lots of people, because everyone will notice many different things that you couldn't find on your own. Being receptive is synonymous with being open to other's points of view.
But even though I take in what I can, I don't accept all points of view, at least on subjects that I care about (on subjects I don't care about, I'm just like, whatever. It's just data to me. When school lets out, I'm going to immerse myself in the library for a little bit everyday, not so much to learn but to expose myself to lots of subjects and get my curiousity up about things. Connie's suggestion was to broaden myself).
In talking about martial arts, as an example of a subject I care about, whenever someone presents a new style that they invented, I always think to myself, "what styles has this guy trained in? Where did he receive the training? Who trained him? How long in each style? What was the training like? What exactly did he do in the training? What's the basic idea of this new style? etc."
People are inventing new styles every month it seems. Americans are trying to patent new "street styles" and "street self-defense moves", yatta yatta. I like to skim across them and see what's out there, but most of it is all for the sake of money. The actual heart of street fighting is simple and direct. People like to hype it up and make it seem like it's a big deal, when it really isn't. They're just after your money, most of them. "Give me your money, and you can be a master just like me!" Then there's a picture of some old guy with a grimacing face, punching the crap out of some punk. I think to myself, "yeah, okay."
That's part of the reason why I like older styles. There's less of "I want your money" and more of "be a part of an ancient tradition." There are pros and cons to this as well. For one thing, there's not a lot of room for individuation, but that's the point of training in an ancient style like karate or aikido or kung fu. You practice a set of moves, and learn efficiency and economy of energy, or whatever it is they're trying to teach you. The lessons aren't in teaching you how to fight, they're in teaching you how to fight "well", and that definition of "well" varies from style to style. Kenpo would say "well" is generating maximum power. Aikido would say "well" is controlling the flow of energy. Wado-ryu would say "well" is using minimum movement and being simple.
This is an example of manyness, and the oneness comes from the heart of fighting, which is the here-and-now of "fight or flight." You don't have time to think about what you're going to do, you're not focused on anything except the guy you're fighting, that's the oneness. You just flow.
On the other hand, you have to look at what stance the guy is in, what is his mood (aggressive or defensive), what counters would be effective, what moves would work and what won't, should I concentrate on striking or should I take him to the ground, does this guy have friends, and importantly how can I get away, etc. That's manyness (it's also called tactics).
Bruce Lee said that a martial artist needs to have disciplined himself in using many techniques that work, but also to have developed a free-flow zen of fighting, so that you can react instantly. Don't think, just fight. "No-mind." That's oneness.
A Jeet Kune Do practictioner, in theory, (paraphrasing old Bruce here) can be walking down the street. All of a sudden he's attacked by some guy. Without thinking, the JKD man flows in and beats him, and keeps walking. Fighting is simply a part of his walk, and he doesn't need to give it any thought.
I think that's just WILD! (Small tangent)I don't agree with Bruce Lee there. I mean, I see what he's saying. If all you wanted to do was protect yourself, than that's a good way to go, but my training in Wado-ryu would say the first thing you should do is run away, or talk the other guy down. Morals. There's usually a peaceful solution. Fighting is the last option for me.

Now in all things, there must be balance.
If you had just oneness, and you didn't think about all that stuff, not even a little bit, you can easily be beaten with just one counter. I fought a guy who was like this at my old sensei's dojo. He refused to learn all the techniques, and just went completely street fighter on everyone. It was intimidating. He'd just leap towards you with punch and kick. For a long time, he won a lot. I didn't know how to counter it, until sensei taught me the defensive side-kick. With that one counter, and a little bit of thought, the guy couldn't beat me. He hadn't thought about how his tactic might one day fail him, and so he COULDN'T CHANGE.
Likewise, if you had just manyness and no oneness, you wouldn't be able to fight. You'd be so anxious about picking the right counter, that you'd run out of time and not be able to act. Unfortunately, I've only heard of people like this, not met them. An example would be a guy who got punched in the face and then asked the aggressor to wait while he "got in his stance."
Anyway, that was an example of a subject I care about. It's both simple and complicated at the same time. That's what I think oneness and manyness is.

Now, how does this apply to Mentor? You should be learning and applying all of the techniques we're teaching you: meditation and clearing process are the two main ones. Along with this, learning all of the models, learning techniques for dealing with others, such as mirroring and body language. Reading all the books and seeing all the movies. Learning philisophies, cultures, and religions of other peoples. Etc.
This is the manyness of Mentor.
And at the same time, we want you to "unbrainwash" yourselves and become autonomous. This involves pulling up feelings and working on them, as well as analyzing your inherent assumptions.
This is the oneness of Mentor. It's hard to explain of paper, but it's the simple act of growing yourself. You are here to grow, plain and simple.
My point is that you can't do one without doing the other. You've got to be balanced. Each of these things will supplement the other, adding to each other. So if you're focused on just rote memorizing all the of things, try also to keep in mind the self and the simplicity of why you are here. My Mom would tell me, "don't lose sight of the goal." Another way of saying it is "walk the path." Anyone can learn all of this stuff, but if they aren't "deep", than they're not going to grow.
And for those of you who are motivated and wanting to grow, already digging up feelings and analyzing assumptions, trying to "walk the path", keeping sight of "the goal", try to learn all the different things and practice them. Being "deep" without guidance and focus is utterly useless and a waste of time (I think this is Cor's and my issue). See the movie, Crouching Tiger, Hidden Dragon. Love that movie! Wow! "I can teach you to fight with the Green Destiny, but first you must learn to hold it in stillness."

Love.
Scott


\section{Quotes}

\section{Works Cited}

\section{Further Reading}


\bibliographystyle{mla}		% the style you want to use for references.
\bibliography{mr,refs}				% the files containing all the articles and books you ever referenced.
