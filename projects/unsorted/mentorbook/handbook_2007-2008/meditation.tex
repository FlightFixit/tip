\chapter{Meditation (Craig)}

\begin{authors}
	Craig Schuff
\end{authors}

\section{Introduction}

Mentor is a program of training for the mind. Training is a bit different from what we may normal associate with traditional education. A normal course may have an informative title and a syllabus that lists what the student will be expected to know by the end. In class students are given lectures and demonstrations a set of knowledge and skills which are to be reinforced by homework. Later, testing seeks to confirm that a student can adequately repeat what the teacher has presented in class. Memorization and regurgitation is often the surest way of passing in this system which is developed in order to shove mass quantities of data through variously shaped holes.


Our initial introduction to the concept of training often comes to us through our bodies rather than our minds. We believe that we study new things to fill our minds, but we train to expand the ability of our bodies. So while the mind is treated as an entity of minimal flexibility, the body is taught to always exceed its limitations. If we can crawl we push ourselves to walk. It isn't a chore, it's a delight. We see everyone around us walking and so it is less a question of if but rather when. **Perhaps when adults are not around babies ponder their failures and philosophize about whether or not people should walk, but if so they deal poorly with peer pressure and resume their attempts in the presence of the walking class.** Running follows quickly long before walking is mastered. A little later in life we may return to running but from a very different perspective.

Our goal now is not to learn how to run, but how to run better. In the pursuit of this goal a coach will challenge us to look at running from all new directs. If we want to master the skill running we have to master other parts of our body that are interconnected with it. We learn to warm up and stretch. We run at various distances and at various speeds. We run on track, grass, and sand. We study nutrition and learn how to best fuel our body. We learn new ways to breath. We learn to watch ourselves and recognize what our body is communicating to our mind. With experience we can accurately name the state of our body. The general label of tired may be further broken down into feeling sleepy, out of breath, or dehydrated. With familiarity we can identify pain and know when to stop or when to push through. Each little thing builds off the others to make a healthy body and then a mature athlete. Each basic part of the whole is examined and revisited many times as flaws are examined and form is improved. The body is changed in its ability as new movements are practiced to the point that they become habit. This process is called training, and it is most productive with a committed trainee and an experienced instructor.


How is this relevant to our minds?


********

The word meditation covers a wide range of ideas and disciplines spanning ancient cultures, world religions, and modern science. A parishioner counting beads and chanting prayers, a monk seated quietly with legs crossed in lotus, and a student of the martial arts moving through careful routine can all accurately claim to practice meditation. Despite their differences each form shares some commonality in its basic goals and challenges.


No matter the meditation of choice, the initial task is to develop focus. Focus is the ability of directed attention. It is taken for granted that we control our thoughts, but the first few seconds of trying to really be still reveals how limited our control of attention actually is.


A common theme of meditation is dealing with distraction and choosing when and where to focus our minds. Mastery of distraction is invaluable. If distraction cannot sway us from our path then each line of thought we take is a choice. When something new comes into our attention we can then decide if it is worth time now, later, or not at all. Although simple in concept, actually mastering focus requires commitment and discipline. Furthermore, because focus and distraction are so basic to how we interact with our world, improving in this area changes the quality of everything we do.


Everywhere we go pictures, slogans, and music fight for limited space in our mind. Flashy advertising is designed to pull us out of our previous thoughts and hold on. When we sit down to study and our minds wander and we lose focus. Our reading slows down, retention suffers, and we may miss important details.


\section{Basic Techniques}

Our concern is a practice called just sitting.




To start find a sitting position that is relatively comfortable. While a traditional lotus position may look impressive it will be more of a distraction than aid without a certain amount of flexibility. A cross legged position or simply sitting on the edge of a chair will serve just as well. The important thing is to sit with your back straight, shoulders back and head level. This will keep you windpipe open, and preempt the general drowsiness that can accompany a slouched posture.




Breathe in through your nose and out through your mouth in a deep and steady manner. A good breath can be felt down in your gut, expanding your stomach rather than your chest. A few deep breathes in a straight posture will have an immediate affect on your brain and body. At the start of your meditation they also serve as a marker point. With these breathes we drop what ever else has been on our mind and refocus on our meditation here and now. As you progress your breath will reduce in volume as your mind slows down and your body requires less oxygen. A slight tingling in your lips may indicate hyperventilation, a condition where your body is taking in too much oxygen. This is a sign that you can further reduce your air intake.



For now close your eyes and visualize a white disk. Then let it grow bigger until it fills your field. As thoughts, feelings, and distractions arise simply notice them and gently let them pass out of your mind.


�meditation is very much like training a puppy. You put the puppy down and say "Stay," Does the puppy listen? It gets up and it runs away. You sit the puppy back down again. "Stay." And the puppy runs away over and over again. Sometimes the puppy jumps up, runs over, and pees in the corner or makes some other mess. Our minds are much the same as the puppy, only they create even bigger messes. In training the mind, or the puppy, we have to start over and over again.

�frustration comes with the territory. Nothing in our culture or our schooling has taught us to steady and calm our attention. One psychologist has called us a society of attentional spastics. Finding it difficult to concentrate, many people respond by forcing their attention on their breath or mantra or prayer with tense irritation and self-judgment, or worse. Is this the way you would train a puppy? Does it really help to beat it? Concentration is never a matter of force or coercion. You simply pick up the puppy again and return to reconnect with the here and now.

-Jack Kornfield

. Start by doing this for periods of 5 minutes a few times a day and work up to doing at least 20 minutes once a day. It will be difficult at first, which is why we are doing it.


-Craig

\begin{quotation}
The hallmark of the meditation process is in being ``here'' and not
``there.'' Indeed, the focal point of continuity is in being here at all
times. The famous message of Ram Dass to ``Be here now'' is what results
when one is adept in this practice. It is laborious in that it requires
great perseverance---we are up against lifelong patterns---but it is a
major enlightenment practice because it can break through our basic
conditioning. The secret of success in continuity practice is to
eliminate any sense of failure. From the moment we begin, we are
successful. The only measure of success is this moment, right now. Are
we here? If we are here, our practice is perfect. The fact that we have
just returned from out yonder, or that we might take off again in a few
seconds, is not relevant. Without practice, we would always be spaced
out. We would rarely experience being here. Thus, each moment we are
able to break the pattern, we have succeeded.
\begin{flushright}
	David A. Cooper
\end{flushright}
\end{quotation}

\section{Quotes}

\section{Works Cited}

\section{Further Reading}

\bibliographystyle{mla}		% the style you want to use for references.
\bibliography{mr,refs}				% the files containing all the articles and books you ever referenced.

