\chapter{Paradox (Connie)} \label{ch:paradox}

\begin{authors}
	by people who eventually send me stuff
\end{authors}

\section{Paradox in Eliot: from Craig}

T.S. Elliot
From Little Gidding

"There are three conditions which often look alike
Yet differ completely, flourish in the same hedgerow:
Attachment to self and to things and to persons, detachment
From self and from things and from persons; and, growing between them,
indifference
Which resembles the others as death resembles life,
Being between two lives�unflowering, between
The live and the dead nettle."

First state:  attachment to self and to things and to persons
Second state: detachment from self and from things and from persons

Resolution: growing between them, indifference
Which resembles the others as death resembles life

The Paradox is the simultaneous realization of the two states which
seemingly exclude each other.  This resolution resembles the original
statements not at all.  The two opposites held in tension creates a
force which is entirely new and unexpected.

\subsubsection{Journal: John Visel, 31 July 2007}

Here's a poem I wrote a two years ago.  In retrospect, it's something
of a baby picture, but many times, I still think this way.  I think I
was trying Rumi's ideas on for size- I didn't really ``know'' them on any
significant level yet, but looking back, it was a growth step to
pretend his ideas were something I embodied.  That sort of thing is the
spiritual equivalent of children playing ``house'' or ``school'' or
``grownups.''  It's part of how we learn in any arena.

John

***

\medskip

\noindent The mystery used to have an answer\\
It was just elusive, and I could never find it.\\
Now, the answer to the mysteries of life\\
are not what I'm seeking as much,\\
More just to develop a relationship with them.\\
The mystery is like a spiritual figure\\
Or a sensual desire, of something\\
we know next to nothing about,\\
only that it's attractive\\
The artists\\
Have this on a personal\\
and spiritual level.\\
Beethoven's Immortal Beloved\\
(not to mention this happening\\
all over spirituality/mysticism.)\\
Someone who is far away, is prescribed\\
godlike status, yet whose foundation\\
or real definition, we don't really know.\\
This happens all over the place\\
Love from Afar\\
Dante's Beatrice,\\
Petrarch's Laura,\\
Bocaccio's Fiammetta,\\
These loves are different than normal ones\\
We know they're not attainable\\
in any earthly form,\\
Yet they're divinely attractive\\
What a tough concept to get one's\\
head around!\\

\medskip

\noindent The reward of this kind of mystery\\
is in the striving for it.\\
To achieve the ``answer'' to this mystery,\\
one must transform.\\
The answer may lie in the person you become\\
once you've struggled with this\\
and it has hurt you\\
in a big-context sort of way.\\

\medskip

\noindent Perhaps we're not looking off in to\\
some faraway place for this love\\
Perhaps we're looking in at the divine\\
or god-part of ourselves.\\
Perhaps that's why we can't find\\
and answer in the form that we want it.\\
Perhaps that's why we so much think\\
That it's just one thing.\\

\medskip

\noindent Much of the pain we've been through\\
has been because we're looking for an answer\\
in a form that we've long since outgrown\\
And when you finally realize this,\\
and learn to just be with the mystery,\\
It doesn't feel like you're getting answers\\
anymore.  Yet you're content.\\
In reality, you've come to terms with the mystery,\\
and that may be the answer itself.\\
In that sense, most of us are barking\\
up the wrong tree,\\
Or climbing a lifelong ladder,\\
Only when we reach the top, tired, and frustrated,\\
We find that we've put our ladder\\
up against the wrong building.\\

\medskip

\noindent Live with the mysteries\\
Myteries are a condition,\\
not a question with an answer.\\
Mysteries are a space in which to put yourself,\\
not a question with an answer.\\

\medskip

\noindent Most of us assume that there is just one answer\\
Which is fascinating. Our minds are instinctively\\
steering us toward one-ness\\
Why must there be just one?\\
Perhaps if we were content with many answers\\
for the same question,\\
it wouldn't be quite so intriguing.\\
And we wouldn't strive\\
quite so hard.\\
The striving, after all, I think,\\
is how we progress down this path.

\section{Quotes}

\section{Works Cited}

\section{Further Reading}

\bibliographystyle{mla}		% the style you want to use for references.
\bibliography{mr,refs}				% the files containing all the articles and books you ever referenced.
